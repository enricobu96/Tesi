% !TEX encoding = UTF-8
% !TEX TS-program = pdflatex
% !TEX root = ../tesi.tex

%**************************************************************
% Sommario
%**************************************************************
\cleardoublepage
\phantomsection
\pdfbookmark{Sommario}{Sommario}
\begingroup
\let\clearpage\relax
\let\cleardoublepage\relax
\let\cleardoublepage\relax

\chapter*{Sommario}

Il seguente documento descrive il tirocinio che ho svolto presso la sede di Padova dell'azienda \textbf{Sync Lab s.r.l.}, d'ora in poi \textit{"Sync Lab"} o \textit{"l'azienda"}, nel periodo che va dal 31/08/2020 al 23/10/2020. Lo stage ha avuto una durata complessiva di circa \textbf{300 ore}, ed è stato supervisionato dal mio tutor esterno, Andrea Giunta, e dal mio relatore, prof. Tullio Vardanega.\\
Lo scopo dello stage è stato quello di studiare e utilizzare il \textit{framework Ionic}, tecnologia che permette la realizzazione di applicazioni \textit{mobile} ibride a partire da codice \textit{JavaScript} o \textit{TypeScript}, per il \textbf{\textit{porting}} di un applicativo web preesistente su dispositivi mobili. \\
La migrazione verso dispositivi mobili che ho effettuato si è basata sulla componente \textit{front-end} dell'applicativo \textit{SyncTrace}, sviluppata da altri studenti durante i precedenti tirocini e completata ufficialmente il 28/08/2020; questa componente è stata sviluppata tramite il \textit{framework Angular} in accoppiata al linguaggio \textit{TypeScript}. \\
L'\textbf{obiettivo} di questo applicativo è inquadrabile nell'ambito del \textit{contact tracing} per la prevenzione di infezione da \textit{Sars-CoV2}: esso permette infatti il monitoraggio di persone infette, e a rischio infezione, da parte di medici ed esercenti.\\
Il presente documento è suddiviso nei seguenti capitoli:
\begin{itemize}
  \item \hyperref[cap:contesto-aziendale]{\textbf{Capitolo 1}}: presentazione dell'azienda, comprendente i processi messi in atto dall'azienda per soddisfare il cliente, la metodologia di sviluppo, le tecnologie di interesse e un approfondimento sulla propensione dell'azienda all'innovazione;
  \item \hyperref[cap:progetto-contesto-aziendale]{\textbf{Capitolo 2}}: descrizione della proposta di \textit{stage}, comprendente gli obiettivi e i vincoli a cui mi sono dovuto attenere e la motivazione della scelta. Descrive inoltre la fase iniziale del mio tirocinio, corrispondente alla formazione sulle tecnologie e sul precedente progetto, punto di partenza del mio lavoro;
  \item \hyperref[cap:progetto-stage]{\textbf{Capitolo 3}}: descrizione dettagliata del progetto di \textit{stage}, ossia delle attività di sviluppo, verifica, validazione e collaudo che ho effettuato per ottenere il prodotto finale;
  \item \hyperref[cap:valutazioni-retrospettive]{\textbf{Capitolo 4}}: valutazione retrospettiva dello \textit{stage}, contenente il soddisfacimento degli obiettivi e le competenze professionali che ho maturato.
\end{itemize}

In questo documento ho utilizzato le seguenti \textbf{convenzioni tipografiche}:
\begin{itemize}
  \item I termini in lingua straniera e i termini propri sono evidenziati dall'utilizzo del corsivo;
  \item Piccole sezioni di testo possono essere riportate in grassetto, al fine di evidenziarne l'importanza nel discorso;
  \item Tutte le tabelle e le immagini riportano un numero progressivo, una descrizione e, se provengono da una risorsa esterna, la fonte.
\end{itemize}


%\vfill
%
%\selectlanguage{english}
%\pdfbookmark{Abstract}{Abstract}
%\chapter*{Abstract}
%
%\selectlanguage{italian}

\endgroup

\vfill
