% !TEX encoding = UTF-8
% !TEX TS-program = pdflatex
% !TEX root = ../tesi.tex

%**************************************************************
\chapter{Il progetto di stage}
\label{cap:progetto-stage}
%**************************************************************

\section{Analisi dei requisiti}

Sezione contenente l'analisi dei requisiti del prodotto software. In questa sezione verranno riportati anche i diagrammi dei casi d'uso dell'applicativo e, possibilmente, i \textit{mockup} dell'applicazione creati prima di cominciare le attività di progettazione e codifica.

%**************************************************************

\section{Progettazione}

Sezione contenente le scelte progettuali che ho preso, insieme al tutor esterno, per lo sviluppo dell'applicativo. Poiché gran parte delle scelte è stata presa durante il tirocinio antecedente al mio, questa sezione non avrà una grande estensione; nonostante questo, alcune scelte progettuali sono state prese da me, e ritengo quindi giusto riportarle.

%**************************************************************

\section{Codifica}

Sezione contenente una descrizione dell'attività di codifica applicata al mio progetto.

%**************************************************************

\section{Verifica}

Breve introduzione all'attività di verifica svolta, in cui riassumerò le modalità adottate nel caso del mio progetto software e gli strumenti utilizzati per il \textit{testing}.

\subsection{Analisi statica}

Descrizione dell'utilizzo dell'analisi statica per il controllo di conformità del codice agli standard. In questa sottosezione parlerò anche degli strumenti utilizzati per effettuare l'analisi statica del codice e dei risultati ottenuti in quest'ambito.

\subsection{Test di unità}

Sezione riportante tutte le informazioni riguardanti i test di unità implementati e i risultati delle misure.

%**************************************************************

\section{Validazione e collaudo}

Sezione contenente l'analisi delle attività di validazione eseguite. La sezione sarà divisa in due sottosezioni non numerate: la prima riguarderà la validazione, e quindi conterrà l'analisi dei requisiti soddisfatti; la seconda conterrà l'analisi del collaudo. In quest'ultima sottosezione parlerò anche della containerizzazione del back-end di SyncTrace, poiché è stata un'attività fondamentale per poter eseguire il collaudo dell'applicazione su un dispositivo mobile.

%**************************************************************

\section{Risultati ottenuti}

Sezione contenente una visione d'insieme dei risultati ottenuti, ossia un riassunto di quanto sviluppato, verificato, validato e documentato.
