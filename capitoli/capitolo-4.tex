% !TEX encoding = UTF-8
% !TEX TS-program = pdflatex
% !TEX root = ../tesi.tex

%**************************************************************
\chapter{Valutazioni retrospettive}
\label{cap:valutazioni-retrospettive}
%**************************************************************

\section{Soddisfacimento degli obiettivi}
Una volta conclusi i processi di sviluppo e documentazione ho analizzato insieme al tutor aziendale gli obiettivi fissati ad inizio \textit{stage}, al fine di individuarne il grado di soddisfacimento e poter effettuare un'analisi a posteriori di cosa avrei potuto gestire meglio. In \hyperref[tab:grado-soddisfacimento]{Tabella 4.1} riporto gli obiettivi fissati, il loro grado di soddisfacimento e una breve nota che sintetizza questo soddisfacimento. \\

\begin{table}[h]
  \label{tab:grado-soddisfacimento}
  \begin{center}
\begin{tabular}{llp{8cm}}
\textbf{Obiettivo}         & \textbf{Risultato}               & \textbf{Note}            \\ \hline
\multicolumn{1}{|l|}{\texttt{O-O1}} & \multicolumn{1}{l|}{Soddisfatto} & \multicolumn{1}{p{8cm}|}{Ho correttamente acquisito le competenze su tutti i linguaggi di programmazione, i \textit{framework} e gli strumenti di supporto previsti dal piano di lavoro} \\ \hline
\multicolumn{1}{|l|}{\texttt{O-O2}} & \multicolumn{1}{l|}{Soddisfatto} & \multicolumn{1}{p{8cm}|}{Ho correttamente seguito il cronoprogramma in autonomia, arrivando a impiegare meno tempo di quanto previsto} \\ \hline
\multicolumn{1}{|l|}{\texttt{O-O3}} & \multicolumn{1}{l|}{Soddisfatto} & \multicolumn{1}{p{8cm}|}{Ho portato a termine le implementazioni previste con una percentuale di superamento pari al 100\%, arrivando a sviluppare anche funzionalità aggiuntive} \\ \hline
\multicolumn{1}{|l|}{\texttt{O-D1}} & \multicolumn{1}{l|}{Soddisfatto} & \multicolumn{1}{p{8cm}|}{\textit{Come sopra}} \\ \hline
\multicolumn{1}{|l|}{\texttt{O-D2}} & \multicolumn{1}{l|}{Soddisfatto} & \multicolumn{1}{p{8cm}|}{Ho effettuato la corretta containerizzazione della componente \textit{back-end} del progetto \textit{SyncTrace}, rendendo disponibili anche le interfacce grafiche per la gestione dei microservizi e del \textit{database}} \\ \hline
\multicolumn{1}{|l|}{\texttt{O-F1}} & \multicolumn{1}{l|}{Soddisfatto} & \multicolumn{1}{p{8cm}|}{Ho effettuato una corretta progettazione che ha reso l'applicazione sviluppata usabile, \textit{responsive} e \textit{adaptive}} \\ \hline
\multicolumn{1}{|l|}{\texttt{O-F2}} & \multicolumn{1}{l|}{Soddisfatto} & \multicolumn{1}{p{8cm}|}{Ho implementato correttamente la lettura del \textit{QR Code} direttamente della fotocamera. Per ragioni di compatibilità con le scelte di ogni utente, ho inoltre reso disponibile anche l'inserimento manuale del codice} \\ \hline
\end{tabular}
\end{center}
\caption{Grado di soddisfacimento degli obiettivi di \textit{stage}.}
\end{table}

Mi sono inoltre impegnato a consegnare tutti i prodotti richiesti dall'azienda; questi sono riassunti dalla \hyperref[tab:prodotti-sviluppati]{Tabella 4.2}.

\begin{table}[h]
  \label{tab:prodotti-sviluppati}
  \begin{center}
  \begin{tabular}{lp{9cm}}
  \textbf{Prodotto}                        & \textbf{Note}                                                                                                                                                                                                                                                                                                                                    \\ \hline
  \multicolumn{1}{|l|}{Codice}             & \multicolumn{1}{p{9cm}|}{Durante tutto il processo di sviluppo ho utilizzato il repository aziendale riservato agli stage. Ho quindi consegnato ufficialmente il codice effettuando il \textit{merge} dei \textit{branch} da me creati sul \textit{master branch}}                                                                                                               \\ \hline
  \multicolumn{1}{|l|}{Documentazione}     & \multicolumn{1}{p{9cm}|}{La documentazione che ho consegnato consiste nella documentazione tecnica. Questa è integrata nei \textit{file} contenenti il codice, e ho provveduto ad automatizzarne la visualizzazione, con il \textit{tool Compodoc}, tramite la creazione di uno \textit{script npm}}                                                                             \\ \hline
  \multicolumn{1}{|l|}{Containerizzazione} & \multicolumn{1}{p{9cm}|}{Per quanto concerne la containerizzazione, ho rilasciato sul \textit{repository} precedentemente citato due \textit{file yml} che, eseguiti tramite \textit{docker compose}, costruiscono e mettono a disposizione rispettivamente la componente \textit{back-end} di SyncTrace e le interfacce grafiche per la gestione dei microservizi \textit{Spring} e del \textit{database}} \\ \hline
  \end{tabular}
\end{center}
\caption{Prodotti sviluppati e messi a disposizione dell'azienda.}
  \end{table}

In conclusione posso ritenermi soddisfatto di quanto svolto, poiché sono riuscito a completare tutti i prodotti richiesti nel tempo prestabilito e con il massimo grado di soddisfacimento degli obiettivi. \\
Ho riscontrato soltanto due problematiche: una riguardante tutto lo \textit{stage} e una riguardante soltanto l'attività di analisi dei requisiti. La prima è che, purtroppo, ho dovuto svolgere gran parte del mio lavoro in \textit{smart working}, non potendo accedere alla sede per la maggior parte delle giornate lavorative. Per quanto sia comunque riuscito a organizzarmi al meglio per lavorare da casa, infatti, avrei preferito frequentare maggiormente l'azienda, poiché quando mi sono recato di persona ho potuto constatare un clima costruttivo e di collaborazione, a mio parere importante per la mia crescita lavorativa. Naturalmente, questo non è imputabile all'azienda ma al periodo attuale; giudico infatti responsabile questa scelta da parte dell'azienda. \\
La seconda problematica che ho riscontrato riguarda l'analisi dei requisiti. Sebbene il prodotto su cui il mio progetto è basato offrisse una documentazione tecnica estremamente dettagliata e utile allo scopo, parte dell'analisi dei requisiti non è stata caricata sul \textit{repository} aziendale, e questo ha rallentato lievemente la mia attività di analisi dei requisiti. \\
Per quanto concerne gli obiettivi personali posso, anche in questo frangente, dirmi pienamente soddisfatto del percorso che ho svolto. Era infatti mia intenzione approfondire l'ambito dello sviluppo di applicativi \textit{mobile} e utilizzare i linguaggi \textit{JavaScript} e \textit{TypeScript}, in accoppiata ad almeno un \textit{framework} di attuale utilizzo. Dallo \textit{stage} sono riuscito ad ottenere entrambi, poiché ho sviluppato un'applicazione per dispositivi mobili proprio tramite l'utilizzo di \textit{TypeScript} e dei due framework \textit{Angular} e \textit{Ionic}.


%**************************************************************

\section{Bilancio formativo}

\subsection{Maturazione professionale}

Lo \textit{stage} che ho effettuato mi ha portato a sviluppare diverse competenze, sia di natura tecnica che professionale. \\
Per quanto riguarda le \textbf{tecnologie} che ho appreso durante il mio periodo in azienda posso citare anzitutto quelle che riguardano \textit{JavaScript} e tutto ciò che ne deriva: ho infatti approfondito il linguaggio \textit{TypeScript}, arrivando ad averne una conoscenza abbastanza estesa, e  soprattutto ho appreso il funzionamento dei \textit{framework Angular} e \textit{Ionic}. Di questi ultimi due sono arrivato ad avere una conoscenza di buon livello, applicabile anche al mio futuro lavorativo, poiché ho effettuato la codifica del prodotto software in maniera paragonabile allo sviluppo di un applicativo reale e vendibile da un'azienda. \\
Ho inoltre studiato e utilizzato correttamente \textit{Docker}, strumento sempre più utilizzato dalle aziende per la grande scalabilità che offre durante il processo di sviluppo. Ho infine conosciuto, seppur superficialmente, altre tecnologie molto utilizzate odiernamente dalle aziende; tra queste posso citare il linguaggio \textit{Java}, il \textit{framework Spring} e il \textit{database} relazionale \textit{PostgreSQL}. \\
Per quanto riguarda le competenze di \textbf{natura professionale}, ho potuto toccare con mano il funzionamento della metodologia \textit{Agile}, che in Sync Lab si realizza con l'adozione del metodo \textit{Scrum}; anche questo modello di sviluppo è sempre più utilizzato dalle aziende moderne, poiché permette molta flessibilità permettendo al contempo di raggiungere alti livelli di produttività e qualità. \\
Un'ultima competenza che ho acquisito è definibile con il termine, generico ma corretto, \textit{soft skills}; in questo rientrano i rapporti con i colleghi e con i responsabili, l'organizzazione reale del lavoro e la suddivisione dei compiti all'interno dell'azienda.


\subsection{Rapporto tra università e lavoro}

Come ho esposto a inizio documento, lo \textit{stage} è una parte fondamentale del percorso formativo di uno studente; questo periodo, infatti, permette di uscire dagli schemi vissuti durante i precedenti anni di università, potendo toccare con mano la vita lavorativa all'infuori di essa. Lo \textit{stage} e lo studio accademico sono, a mio parere, complementari, poiché uno \textit{stage} non può essere effettuato al meglio senza il percorso formativo accademico che lo precede, e quest'ultimo non può dirsi concluso finché non viene effettuato uno \textit{stage}. \\
Ritengo che la formazione accademica mi sia risultata utile principalmente per la \textit{forma mentis} che mi ha dato; più precisamente, è mia opinione che una componente fondamentale, necessaria allo svolgimento di un tirocinio, sia l'attitudine al lavoro e al rispetto delle tempistiche previste. In questo vedo un'analogia soprattutto con il corso di Ingegneria del Software, più nello specifico con il progetto didattico: ritengo infatti che la conoscenza dei processi e delle attività che portano alla realizzazione di un prodotto software, trattati durante questo corso, sia stata di grande aiuto nel comprendere i meccanismi che un'azienda adotta per portare a compimento il lavoro. \\
Per quanto riguarda la conoscenza delle tecnologie, invece, ho riscontrato una carenza da parte del percorso universitario che ho affrontato; avendo sviluppato un prodotto tramite l'utilizzo di tecnologie pensate per il web, non ho potuto non fare un parallelismo con quanto studiato durante il corso di laurea. Sebbene sia vero che le tecnologie, soprattutto in questo ambito, sono in continua evoluzione e quindi risulta difficile studiarne una che non diventi obsoleta in relativamente poco tempo, ritengo che sarebbe comunque stato utile approfondire lo sviluppo tramite \textit{framework} durante il mio percorso accademico. Nonostante questo, ritengo che effettuare un tirocinio abbia anche lo scopo di apprendere tecnologie all'avanguardia utilizzate dalle aziende; in questo contesto, quindi, si conferma quanto ho precedentemente affermato, e non posso quindi che ritenermi pienamente soddisfatto dell'intero percorso formativo che ho svolto.

%**************************************************************
